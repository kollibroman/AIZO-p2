\documentclass{article}
\usepackage[utf8]{inputenc}
\usepackage[T1]{fontenc}
\usepackage[polish]{babel}
\usepackage[a4paper, margin=1in]{geometry}
\usepackage{multicol}
\usepackage{amsmath}
\usepackage{gensymb}
\usepackage{graphicx}
\usepackage{multirow}
\usepackage{array}
\usepackage{caption}
\usepackage{float}
\usepackage{hyperref}
\usepackage{karnaugh-map}
\usepackage{tikz}
\usepackage[linesnumbered,ruled,vlined]{algorithm2e}

\begin{document}

\begin{multicols}{3}
    \begin{figure}[H]
        \includegraphics[scale=0.4]{jpg/DOJEBANE_LOGO_PWR.png}
        \label{fig:enter-label}
    \end{figure}
    
    \begin{figure}[H]
    \end{figure}
    
    \begin{figure}[H]
        \includegraphics[scale=0.4]{jpg/w4n.png}
        \centering
        \textbf{W4N}
        \label{fig:WYDZIAŁ INFORMATYKI I TELEKOMUNIKACJI}
    \end{figure}
\end{multicols}

\vspace{75pt}

\begin{center}
    \textbf{\large Badanie efektywności algorytmów grafowych w zależności od rozmiaru instancji oraz sposobu
reprezentacji grafu w pamięci komputera} 
    \vspace{2pt}
    \hrule
    \vspace{4pt}
    \textbf{\large Algorytmy i złożoność obliczeniowa - Projekt} 
\end{center}

\vspace{75pt}

\begin{center}
    \textbf{Autor: } \\
    Filip Kwiek 280947
\end{center}

\begin{center}
    \textbf{Termin zajęć: } \\
    Wtorek np. 11:15
\end{center}

\begin{center}
    \textbf{Prowadzący: } \\
    Dr. inż. Jarosław Mierzwa
\end{center}

\end{document}